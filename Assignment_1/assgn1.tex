\let\negmedspace\undefined
\let\negthickspace\undefined
\documentclass[journal]{IEEEtran}
\usepackage[a5paper, margin=10mm, onecolumn]{geometry}
%\usepackage{lmodern} % Ensure lmodern is loaded for pdflatex
\usepackage{tfrupee} % Include tfrupee package

\setlength{\headheight}{1cm} % Set the height of the header box
\setlength{\headsep}{0mm}     % Set the distance between the header box and the top of the text

\usepackage{gvv-book}
\usepackage{gvv}
\usepackage{cite}
\usepackage{amsmath,amssymb,amsfonts,amsthm}
\usepackage{algorithmic}
\usepackage{graphicx}
\usepackage{textcomp}
\usepackage{xcolor}
\usepackage{txfonts}
\usepackage{listings}
\usepackage{enumitem}
\usepackage{mathtools}
\usepackage{gensymb}
\usepackage{comment}
\usepackage[breaklinks=true]{hyperref}
\usepackage{tkz-euclide} 
\usepackage{listings}
% \usepackage{gvv}                                        
\def\inputGnumericTable{}                                 
\usepackage[latin1]{inputenc}                                
\usepackage{color}                                            
\usepackage{array}                                            
\usepackage{longtable}                                       
\usepackage{calc}                                             
\usepackage{multirow}                                         
\usepackage{hhline}                                           
\usepackage{ifthen}                                           
\usepackage{lscape}
\usepackage{tabularx}
\usepackage{array}
\usepackage{float}
\usepackage{multicol}


\newtheorem{theorem}{Theorem}[section]
\newtheorem{problem}{Problem}
\newtheorem{proposition}{Proposition}[section]
\newtheorem{lemma}{Lemma}[section]
\newtheorem{corollary}[theorem]{Corollary}
\newtheorem{example}{Example}[section]
\newtheorem{definition}[problem]{Definition}
\newcommand{\BEQA}{\begin{eqnarray}}
\newcommand{\EEQA}{\end{eqnarray}}
\newcommand{\define}{\stackrel{\triangle}{=}}
\theoremstyle{remark}
\newtheorem{rem}{Remark}

% Marks the beginning of the document
\begin{document}
\bibliographystyle{IEEEtran}
\vspace{3cm}

\title{Matrix theory}
\author{ai24btech11035 - V.Preethika}
\maketitle
\newpage
\bigskip

\renewcommand{\thefigure}{\theenumi}
\renewcommand{\thetable}{\theenumi}

\begin{enumerate}[start=6]
\item If $z=x+iy$ and $\omega=\frac{1-iz}{z-i}$ , then $\abs{\omega}=1$ implies that,in the complex plane
\hfill{(1983 - 1 Mark)}
\begin{enumerate}
\item z lies on the imaginary axis
\item z lies on the real axis
\item z lies on unit circle
\item None of these
\end{enumerate}
\item The points $z_1,z_2,z_3,z_4$ in the complex plane are the vertices of a parallelogram taken in order if and only if
\hfill{(1983 - 1 Mark)}
\begin{enumerate}
\begin{multicols}{2}
\item $ z_1+z_4=z_2+z_3 $
\item $ z_1+z_3=z_2+z_4 $
\item $ z_1+z_2=z_3+z_4 $
\item None of these
\end{multicols}
\end{enumerate}
\item If $a,b,c$ and $u,v,w$ are complex numbers representing the vertices of two triangles such that $c=(1-r)a+rb$ and $w=(1-r)u+rv$,where $r$ is a complex number,then the two triangles
\hfill{(1985 - 2 Marks)}
\begin{enumerate}
\begin{multicols}{2}
\item have the same area
\item are similar
\item are congruent
\item none of these
\end{multicols}
\end{enumerate}
\item If $\omega\brak{\neq1}$ is a cube root of unity and $\brak{1+\omega}^7=A+B\omega$ then A and B are respectively
\hfill{(1995S)}
\begin{enumerate}
\begin{multicols}{4}
\item $0,1$
\item $2,1$
\item $1,0$
\item $-1,1$
\end{multicols}
\end{enumerate}
\item Let $z$ and $\omega$ be two non zero complex numbers such that $\abs{z}=\abs{\omega}$ and $\mathrm{Arg}\ z+\mathrm{Arg}\omega=\pi$, then $z$ equals 
\hfill{(1995S)}
\begin{enumerate}
\begin{multicols}{4}
\item $\omega$
\item $-\omega$
\item $\overline{\omega}$
\item $-\overline{\omega}$
\end{multicols}
\end{enumerate}
\item let $z$ and $\omega$ be two complex numbers such that $\abs{z}\leq1$ , $\abs{\omega}\leq1$ and $\abs{z+i\omega}=\abs{z-i\overline{\omega}}=2$ then z equals 
\hfill{(1995S)}
\begin{enumerate}
\begin{multicols}{4}
\item 1 or i
\item i or -1
\item 1 or -1
\item i or -1
\end{multicols}
\end{enumerate}
\item For positive numbers $n_1,n_2$ the value of the expression $\brak{1+i}^{n_1}+\brak{1+i^3}^{n_1}+\brak{1+i^5}^{n_2}+\brak{1+i^7}^{n_2}$, where $i=\sqrt{-1}$ is a real number if and only if
\hfill{(1996 - 2 Marks)}
\begin{enumerate}
\begin{multicols}{2}
\item $n_1=n_2+1$
\item $n_1=n_2-1$
\item $n_1=n_2$
\item $n_1>0,n_2>0$
\end{multicols}
\end{enumerate}
\item If $i=\sqrt{-1}$ then $4+5\brak{\frac{-1}{2}+\frac{i\sqrt{3}}{2}}^{334} + 3\brak{\frac{-1}{2}+\frac{i\sqrt{3}}{2}}^{365}$ is a real number if and only if 
\hfill{(1999 - 2 Marks)}
\begin{enumerate}
\begin{multicols}{4}
\item $1-i\sqrt{3}$
\item $-1+i\sqrt{3}$
\item $i\sqrt{3}$
\item $-i\sqrt{3}$
\end{multicols}
\end{enumerate}
\item If $\mathrm{Arg}\ z<0$,then $\mathrm{Arg}\ {-z}-\mathrm{Arg}\ z= $
\hfill{(2000S)}
\begin{enumerate}
\begin{multicols}{4}
\item $\pi$
\item $-\pi$
\item $\frac{-\pi}{2}$
\item $\frac{\pi}{2}$
\end{multicols}
\end{enumerate}
\item If $z_1,z_2$ and $z_3$ are complex numbers such that $\abs{z_1}=\abs{z_2}=\abs{z_3}=\abs{\frac{1}{z_1}+\frac{1}{z_2}+\frac{1}{z_3}}=1$, then $\abs{z_1+z_2+z_3}$ is 
\hfill{(2000S)}
\begin{enumerate}
\begin{multicols}{2}
\item equal to 1
\item less than 1
\item greater than 3
\item equal to 3
\end{multicols}
\end{enumerate}
\item Let $z_1$ and $z_2$ be $n^{\text{th}}$ roots of unity which substend a right angle at the origin.Then n must be of the form
\hfill{(2001S)}
\begin{enumerate}
\begin{multicols}{4}
\item $4k+1$
\item $4k+2$
\item $4k+3$
\item $4k$
\end{multicols}
\end{enumerate}
\item The complex numbers $z_1,z_2$ and $z_3$ satisfying $\frac{z_1-z_3}{z_2-z_3}=\frac{1-i\sqrt{3}}{2}$ are the vertices of a triangle which is
\hfill{(2001S)}
\begin{enumerate}
\item of area zero
\item right angled triangle
\item equilateral
\item obtuse-angled triangle
\end{enumerate}
\item For all complex numbers $z_1,z_2$ satisfying $\abs{z_1}=12$ and $\abs{z_2-3-i}=5$, the minimum value of $\abs{z_1-z_2}$
\hfill{(2002S)}
\begin{enumerate}
\begin{multicols}{4}
\item $0$
\item $2$
\item $7$
\item $17$
\end{multicols}
\end{enumerate}
\item If $\abs{z}=1$ and $\omega=\frac{z-1}{z+1}$ (where $z\neq1$), then $\mathrm Re\brak{\omega}$ is
\hfill{(2003S)}
\begin{enumerate}
\begin{multicols}{2}
\item $0$
\item $\frac{-1}{\abs{z+1}^2}$
\item $\abs{\frac{z}{z+1}}\cdot\frac{1}{\abs{z+1}^2}$
\item $\frac{\sqrt{2}}{\abs{z+1}^2}$
\end{multicols}
\end{enumerate}
\item If $\omega\brak{\neq1}$ be a cube root of unity and $\brak{1+\omega^2}^n = \brak{1+\omega^4}^n$, then the least positive value of n is
\hfill{(2004S)}
\begin{enumerate}
\begin{multicols}{4}
\item $2$
\item $3$
\item $5$
\item $6$
\end{multicols}
\end{enumerate}
\end{enumerate}

\end{document}
